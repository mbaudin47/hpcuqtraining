% Copyright (C) 2018 - 2022 - Michael Baudin & Julien Pelamatti

\documentclass[aspectratio=169]{beamer}

%\setbeameroption{hide notes}
%\setbeameroption{show notes}
%\setbeameroption{show only notes}

  \usepackage{graphicx}
\unitlength=1cm
\graphicspath{{./figures/}{../../../Scripts/Metamodel/}}

\usepackage[space]{grffile}

\usepackage{alltt}
\usepackage{euscript}
\usepackage{hyperref}
\usepackage{bbm}

\usepackage{multicol}

\usepackage[utf8]{inputenc}
\usepackage[T1]{fontenc}

\usepackage{listings}
\definecolor{darkgreen}{rgb}{0,0.5,0}
\definecolor{violet}{rgb}{0.5,0,1}

% Python macros
\newcommand{\pyvar}[1]{\texttt{#1}}

% https://tex.stackexchange.com/questions/183149/cant-silence-a-pdftex-pdf-inclusion-multiple-pdfs-with-page-group-error
\pdfsuppresswarningpagegroup=1

% Configuration beamer
\usetheme{Montpellier}
\setbeamertemplate{navigation symbols}{} % Remove navigation
\useoutertheme{infolines}
\setbeamertemplate{theorems}[numbered] 
% Utilise des fonts serif, pour éviter les pb de fonte
\usefonttheme{serif} 
\setbeamertemplate{caption}[numbered]


% Commandes mathématiques
\newcommand{\RR}{\mathbb{R}}
\newcommand{\ZZ}{\mathbb{Z}}
\newcommand{\esp}{\mathbb{E}}
\newcommand{\var}{\mathbb{V}}
\newcommand{\cov}{\textbf{Cov}}
\newcommand{\likelihood}{\mathcal{L}} % Likelihood

% Lettres en gras
\newcommand{\bc}{{\bf c}}
\newcommand{\bT}{{\bf T}}
\newcommand{\bX}{{\bf X}}
\newcommand{\bY}{{\bf Y}}
\newcommand{\bZ}{{\bf Z}}
\newcommand{\bg}{{\bf g}}
\newcommand{\bs}{{\bf s}}
\newcommand{\bx}{{\bf x}}
\newcommand{\by}{{\bf y}}
\newcommand{\bz}{{\bf z}}
\def\bzero{{\bf 0}}

\newcommand{\bindicator}{\mathbbm{1}}
\newcommand{\red}[1]{\textcolor{red}{#1}}

% Symboles mathématiques gras
\def\balpha{\boldsymbol{\alpha}}
\def\bbeta{\boldsymbol{\beta}}
\def\btheta{\boldsymbol{\theta}}
\def\bmu{\boldsymbol{\mu}}
\def\bxi{\boldsymbol{\xi}}
\def\bpsi{\boldsymbol{\psi}}

\renewcommand{\P}{\mathbb{P}}
% From https://github.com/openturns/openturns/blob/master/python/doc/math_notations.sty
\DeclareMathOperator*{\argmin}{argmin}
\DeclareMathOperator*{\argmax}{argmax}

% Binomial coefficient
\newcommand{\choosefun}[2]{
\left(
\begin{array}{@{}c@{}}
#1\\
#2
\end{array}
\right)
}

\newtheorem{counterexample}{Couter-example}
\newtheorem{hypothesis}{Hypothesis}
\newtheorem{hypotheses}{Hypotheses}



\title[PRACE/UQ: Calibration]{
Calibration: deterministic and bayesian methods \\
PRACE training on High Performance Computing in Uncertainty Quantification 
}

\author[Pelamatti]{
J. Pelamatti, M. Baudin}

%%%%%%%%%%%%%%%%%%%%%%%%%%%%%%%%%%%%%%%%%%%%%%%%%%%%%%

  \begin{document}

%%%%%%%%%%%%%%%%%%%%%%%%%%%%%%%%%%%%%%%%%%%%%%%%%%%%%%%%%%%%%%%%%%%%%%%%%%%%%

\begin{frame}
\titlepage
  
\begin{center}
\includegraphics[height=0.2\textheight]{edf.jpg}
\hspace{1cm}
\includegraphics[height=0.2\textheight]{Maison_Simulation_LOGO.jpg}
\hspace{1cm}
\includegraphics[height=0.2\textheight]{PRACE_LOGO.jpg}
\end{center}

\end{frame}

%%%%%%%%%%%%%%%%%%%%%%%%%%%%%%%%%%%%%%%%%%%%%%%%%%%%%%%%%%%%%%%%%%%%%%%%%%%%%

\begin{frame}
\frametitle{Contents}
\tableofcontents
\end{frame}

%%%%%%%%%%%%%%%%%%%%%%%%%%%%%%%%%%%%%%%%%%%%%%%%%%%%%%%%%%%%%%%%%%%%%%%%%%%%%
\section{Deterministic code calibration}

%%%%%%%%%%%%%%%%%%%%%%%%%%%%%%%%%%%%%%%%%%%%%%%%%%%%%%%%%%%%%%%%%%%%%%%%%%%%%

\begin{frame}
\frametitle{Deterministic code calibration}

We consider a deterministic model $g$ to calibrate:

$$
z = g(\bx, \btheta),
$$

where

\begin{itemize}
\item $p \in\NN$ is the number of parameters, 
\item $m \in\NN$ is the number of inputs, 
\item $\bx \in \RR^{m}$ is the input vector;
\item $z \in \RR$ is the output;
\item $\btheta \in \RR^p$ are the unknown parameters of 
      $g$ to calibrate.
\end{itemize}

\end{frame}

%%%%%%%%%%%%%%%%%%%%%%%%%%%%%%%%%%%%%%%%%%%%%%%%%%%%%%%%%%%%%%%%%%%%%%%%%%%%%

\begin{frame}
\frametitle{Introduction}
Let $n \in \NN$ be the number of observations. 
We assume that $m \geq p$, i.e. the problem is over-determined. 
Let $\bx_1, \ldots, \bx_n \in \RR^{m}$ be the observed inputs. 

Therefore, 
$$
z_i = g(\bx_i, \btheta),
$$

for $i=1,...,n$. 
The vector of predictions is $\bz \in \RR^{n}$. 

Let $\bh : \RR^p \rightarrow \RR^{n}$ be the function:
$$
h_i(\btheta) = g(\bx_i, \btheta)
$$
for $i=1, \ldots, n$. 
We write it in the vector form:
$$
\bz = \bh(\btheta)
$$
where $\bh(\btheta) = (h_1(\btheta), \ldots, h_n(\btheta))^T$.

\end{frame}

%%%%%%%%%%%%%%%%%%%%%%%%%%%%%%%%%%%%%%%%%%%%%%%%%%%%%%%%%%%%%%%%%%%%%%%%%%%%%

\begin{frame}
The standard hypothesis of the probabilistic calibration is the following. 
There is a \emph{true}, but unknown, value of the vector $\btheta$, 
denoted by $\btheta^\star$ such that:

$$
\by = \bh(\btheta^\star) + \vect{\varepsilon},
$$

for $i=1,...,n$ where $\vect{\varepsilon}$ is a random
measurement error vector such that:

$$
\esp(\vect{\varepsilon}) = \vect{0} \in \RR^{n}, \qquad 
\cov(\vect{\varepsilon}) = R^\star \in \RR^{n\times n},
$$

where $R^\star$ is an unknown semi-positive definite covariance matrix.

\end{frame}

%%%%%%%%%%%%%%%%%%%%%%%%%%%%%%%%%%%%%%%%%%%%%%%%%%%%%%%%%%%%%%%%%%%%%%%%%%%%%

\begin{frame}
The goal of calibration is to estimate $\btheta$, based on
observations of $n$ inputs
$\bx_1, \ldots, \bx_n$ and the associated $n$
observations of the output $\by$. 

In other words, the calibration process reduces the discrepancy between 
\begin{itemize}
\item  the observations $\by$ and 
\item  the predictions $\bh(\btheta)$. 
\end{itemize}

\vspace{20pt}

Given that $\by$ is the realization of a random
vector, the estimate of $\btheta$, denoted by
$\hat{\btheta}$, is also a random variable. 

The secondary goal of calibration is to estimate the distribution of
$\hat{\btheta}$ representing the uncertainty of the
calibration process.

\end{frame}

%%%%%%%%%%%%%%%%%%%%%%%%%%%%%%%%%%%%%%%%%%%%%%%%%%%%%%%%%%%%%%%%%%%%%%%%%%%%%

\begin{frame}
\frametitle{Example : 2 parameters model}
\begin{figure}
\centering
\includegraphics[width=.7\textwidth]{example.pdf}
\end{figure}

\end{frame}

%%%%%%%%%%%%%%%%%%%%%%%%%%%%%%%%%%%%%%%%%%%%%%%%%%%%%%%%%%%%%%%%%%%%%%%%%%%%%

\begin{frame}
The standard observation model makes the hypothesis that the observation error is Gaussian, and that every observation has an independent error:

$$
\vect{\varepsilon} = \mathcal{N}\left(\bzero, (\sigma^\star)^2 \imat\right)
$$

where $\sigma^\star > 0$ is the (unknown) constant observation error
variance.

\end{frame}

%%%%%%%%%%%%%%%%%%%%%%%%%%%%%%%%%%%%%%%%%%%%%%%%%%%%%%%%%%%%%%%%%%%%%%%%%%%%%

\begin{frame}
\frametitle{Least squares}

The residuals vector is made of the differences between the observations 
and the predictions:

$$
\vect{r}(\btheta) = \by - \bh(\btheta),
$$
for any $\btheta \in\RR^p$. 

When the parameter $\btheta$ is the \emph{true} $\btheta^\star$, the residual 
vector is:
$$
\vect{\varepsilon} = \by - \bh\left(\btheta^\star\right).
$$


The method of least squares minimizes the square
of the Euclidian norm of the residuals \cite{Bjorck1996}:
$$
c(\btheta) 
= \frac{1}{2} \|\by - \bh(\btheta)\|^2 
= \frac{1}{2} \sum_{i=1}^n \left( y_i - h_i(\btheta) \right)^2,
$$
for any $\btheta \in \RR^p$.


\end{frame}

%%%%%%%%%%%%%%%%%%%%%%%%%%%%%%%%%%%%%%%%%%%%%%%%%%%%%%%%%%%%%%%%%%%%%%%%%%%%%

\begin{frame}
The least squares method minimizes the cost function $c$:
$$
\hat{\btheta} 
= \argmin_{\btheta \in \RR^p} \frac{1}{2} \|\by - \bh(\btheta)\|^2.
$$

\vspace{30pt}

The unbiased estimator of the observation noise variance is:

$$
\hat{\sigma}^2 = \frac{\|\by - \bh(\btheta)\|^2}{n - p}.
$$

Notice that the previous estimator is not the maximum likelihood
estimator (which is biased).
\end{frame}

%%%%%%%%%%%%%%%%%%%%%%%%%%%%%%%%%%%%%%%%%%%%%%%%%%%%%%%%%%%%%%%%%%%%%%%%%%%%%

\begin{frame}
\frametitle{Linear least squares}

In the particular case where the deterministic function $\bh$
is linear with respect to the parameter $\btheta$, then the
method reduces to the \textbf{linear least squares}. 

\vspace{15pt}

Let $J \in \RR^{n \times p}$ be the Jacobian matrix made of the
partial derivatives of $\bh$ with respect to
$\btheta$:

$$
J(\btheta) = \frac{\partial \bh}{\partial \btheta}.
$$

\vspace{15pt}

Let $\bmu \in \RR^p$ be a reference value of the
parameter $\btheta$. 

Let us denote by $J=J(\bmu)$ the value of the Jacobian at the reference point
$\bmu$. 

Since the function is, by hypothesis, linear, the
Jacobian is independent of $\btheta$. 


\end{frame}

%%%%%%%%%%%%%%%%%%%%%%%%%%%%%%%%%%%%%%%%%%%%%%%%%%%%%%%%%%%%%%%%%%%%%%%%%%%%%

\begin{frame}
Since $\bh$ is linear, it is equal to its Taylor expansion:
$$
\bh(\btheta) = \bh(\bmu) + J (\btheta - \bmu),
$$

for any $\btheta \in \RR^p$.

The corresponding linear least squares problem is:
$$
\hat{\btheta} 
= \argmin_{\btheta \in \RR^p} \frac{1}{2} \|\by - \bh(\bmu) + J (\btheta - \bmu)\|^2.
$$

The Gauss-Markov theorem applied to this problem states that the
solution is:
$$
\hat{\btheta} = \bmu + \left(J^T J\right)^{-1} J^T ( \by - \bh(\bmu)).
$$

\end{frame}

%%%%%%%%%%%%%%%%%%%%%%%%%%%%%%%%%%%%%%%%%%%%%%%%%%%%%%%%%%%%%%%%%%%%%%%%%%%%%

\begin{frame}
The previous equations are the \emph{normal equations}. 

Notice, however, that the previous linear system of equations 
is not implemented as is, i.e.
we generally do not compute and invert the Gram matrix $J^T J$.

\vspace{30pt}

Alternatively, various orthogonalization methods such as the QR or the
SVD decomposition can be used to solve the linear least squares problem
so that potential ill-conditioning of the normal equations is
mitigated \cite{Bjorck1996}.

\vspace{15pt}

When the problem is rank-deficient, a way to proceed is to truncate 
the SVD \cite{Hansen00thelcurve}. 
\end{frame}

%%%%%%%%%%%%%%%%%%%%%%%%%%%%%%%%%%%%%%%%%%%%%%%%%%%%%%%%%%%%%%%%%%%%%%%%%%%%%

\begin{frame}
This estimator can be proved to be the best linear unbiased estimator, that is, among the unbiased linear estimators, it is the one
which minimizes the variance of the estimator \cite{BinghamFry, Sen1990}.

Assuming that the random observations are gaussian:

$$
\vect{\varepsilon} \sim \mathcal{N}(\vect{0},\sigma^2 {\bf I}).
$$

Therefore, the distribution of $\hat{\btheta}$ is:

$$
\hat{\btheta} \sim \mathcal{N} \left(\btheta,\sigma^2 (J^T J)^{-1} \right).
$$


\end{frame}

%%%%%%%%%%%%%%%%%%%%%%%%%%%%%%%%%%%%%%%%%%%%%%%%%%%%%%%%%%%%%%%%%%%%%%%%%%%%%

\begin{frame}
\frametitle{Non Linear Least squares}

In the general case where the function $\bh$ is non linear
with respect to the parameter $\btheta$, then the resolution involves a non linear least squares optimization algorithm. 

\vspace{20pt}

The difficulty in the nonlinear least squares is that, compared to the
linear situation, the theory does not provide the distribution of
$\hat{\btheta}$ anymore.

There are two practical solutions to overcome this limitation.
\begin{itemize}
\item bootstrap (randomly resample within the observations),
\item linearization (in the neighborhood of $\hat{\btheta}$).
\end{itemize}

\end{frame}

%%%%%%%%%%%%%%%%%%%%%%%%%%%%%%%%%%%%%%%%%%%%%%%%%%%%%%%%%%%%%%%%%%%%%%%%%%%%%

\begin{frame}
\frametitle{Link with likelihood maximization}
Assume that the observation noise is gaussian with zero mean and constant 
variance $\sigma^2$: 
$$
\vect{\varepsilon} \sim \mathcal{N}(\bzero,\sigma^2 \imat),
$$ 
where $\sigma>0$ et $\imat\in\RR^{n\times n}$. 

This implies that the observations are independent. 

The likelihood of the i-th observation is:
$$
\ell(y_i|\btheta,\sigma^2) = 
\frac{1}{\sqrt{2\pi \sigma^2}} 
\exp\left(-\frac{(y_i - h_i(\btheta))^2}{2 \sigma^2}\right)
$$
for $i=1,...,n$. 
\end{frame}

%%%%%%%%%%%%%%%%%%%%%%%%%%%%%%%%%%%%%%%%%%%%%%%%%%%%%%%%%%%%%%%%%%%%%%%%%%%%%

\begin{frame}
Since the observations are independent, the likelihood of the observations is 
the product:
$$
\ell(\by|\btheta,\sigma^2) = 
\prod_{i=1}^n \ell(y_i|\btheta,\sigma^2)
$$
for $i=1,...,n$. 

This implies:
\begin{align*}
\log(\ell(\by|\btheta,\sigma^2)) 
&= -\frac{n}{2} \log(2\pi \sigma^2)
-\frac{\|\by - \bh(\btheta)\|_2^2}{2 \sigma^2}
\end{align*}
for any $\btheta\in\RR^p$ and $\sigma>0$. 

We maximize the likelihood with:
\begin{align*}
\hat{\btheta}
= \argmin_{\btheta\in\RR^p} \frac{1}{2} \| \bh(\btheta) - \by\|_2^2
\end{align*}
and:
\begin{align*}
\hat{\sigma}^2
= \frac{1}{n} \| \bh(\hat{\btheta}) - \by\|_2^2.
\end{align*}

\end{frame}

%%%%%%%%%%%%%%%%%%%%%%%%%%%%%%%%%%%%%%%%%%%%%%%%%%%%%%%%%%%%%%%%%%%%%%%%%%%%%
\section{Bayesian code calibration}

\begin{frame}
\frametitle{Bayesian Calibration}

The bayesian calibration framework is based on two hypotheses 
\cite{Tarantola2005, Asch2016} :

\begin{itemize}
\item The parameter $\btheta$ has a
known distribution, called the \emph{prior} distribution, and denoted by
$p(\btheta)$.
\item The output observations
$\by$ are sampled from a known
conditional distribution denoted by $p(\by | \btheta)$.
\end{itemize}


\vspace{15pt}

For any $\by\in\RR^{n}$ such that $p(\by)>0$,
the Bayes theorem implies that the conditional distribution of
$\btheta$ given $\by$ is:

$$
p(\btheta | \by) = \frac{p(\by | \btheta) p(\btheta)}{p(\by)}
$$

for any $\btheta\in\RR^p$.

\end{frame}

%%%%%%%%%%%%%%%%%%%%%%%%%%%%%%%%%%%%%%%%%%%%%%%%%%%%%%%%%%%%%%%%%%%%%%%%%%%%%
\begin{frame}

The denominator of the previous Bayes fraction is independent of
$\btheta$, so that the posterior distribution is
proportional to the numerator:

$$
p(\btheta | \by) \propto  p(\by | \btheta) p(\btheta).
$$

for any $\btheta\in\RR^p$.

\vspace{15pt}

In the Gaussian calibration, the two previous distributions are assumed
to be Gaussian.

We make the hypothesis that the parameter
$\btheta$ follows a Gaussian distribution:

$$
\btheta \sim \mathcal{N}(\bmu, B),
$$

where $\bmu\in\RR^p$ is the mean of the Gaussian prior
distribution, which is named the \emph{background} and
$B\in\RR^{p \times p}$ is the covariance matrix of the
parameter.

\end{frame}

%%%%%%%%%%%%%%%%%%%%%%%%%%%%%%%%%%%%%%%%%%%%%%%%%%%%%%%%%%%%%%%%%%%%%%%%%%%%%
\begin{frame}
We make the hypothesis that the output observations follow the
conditional gaussian distribution:

$$
\by | \btheta \sim \mathcal{N}(\bh(\btheta), R),
$$

where $R\in\RR^{n \times n}$ is the covariance matrix of the
output observations.
\end{frame}


%%%%%%%%%%%%%%%%%%%%%%%%%%%%%%%%%%%%%%%%%%%%%%%%%%%%%%%%%%%%%%%%%%%%%%%%%%%%%
\begin{frame}
\frametitle{Posterior distribution}

Denote by $\|\cdot\|_B$ the Mahalanobis distance associated with
the matrix $B$ :

$$
\|\btheta-\bmu \|^2_B = (\btheta-\bmu )^T B^{-1} (\btheta-\bmu ),
$$

for any $\btheta,\bmu \in \RR^p$. Denote by
$\|\cdot\|_R$ the Mahalanobis distance associated with the matrix
$R$ :

$$
\|\by-\bh(\btheta)\|^2_R = (\by-\bh(\btheta))^T R^{-1} (\by-\bh(\btheta)).
$$

for any $\btheta \in \RR^p$ and any
$\by \in \RR^{n}$. Therefore, the posterior distribution
of $\btheta$ given the observations $\by$ is :

$$
   p(\btheta|\by) \propto \exp\left( -\frac{1}{2} \left( \|\by-\bh(\btheta)\|^2_R 
   + \|\btheta-\bmu \|^2_B \right) \right)
$$

for any $\btheta\in\RR^p$.
\end{frame}

%%%%%%%%%%%%%%%%%%%%%%%%%%%%%%%%%%%%%%%%%%%%%%%%%%%%%%%%%%%%%%%%%%%%%%%%%%%%%
\begin{frame}
\frametitle{MAP estimator}

The maximum of the posterior distribution of $\btheta$ given
the observations $\by$ is reached at :

$$
   \hat{\btheta} = \argmin_{\btheta\in\RR^p} \frac{1}{2} \left( \|\by - \bh(\btheta)\|^2_R 
   + \|\btheta-\bmu \|^2_B \right).
$$

It is called the \emph{maximum a posteriori} estimator or \emph{MAP}
estimator.
\end{frame}

%%%%%%%%%%%%%%%%%%%%%%%%%%%%%%%%%%%%%%%%%%%%%%%%%%%%%%%%%%%%%%%%%%%%%%%%%%%%%
\begin{frame}
\frametitle{Regularity of solutions of the Gaussian Calibration}

The gaussian calibration is a tradeoff, so that the second expression
acts as a \emph{spring} which pulls the parameter $\btheta$
closer to the background $\bmu$ (depending on the "spring
constant" $B$, meanwhile getting as close a possible to the
observations. 

Depending on the matrix $B$, the computation may
have better regularity properties than the plain non linear least
squares problem.

This is similar to Tikhonov regularization, with identity 
matrix $B$ and zero parameter $\bmu$ \cite{mueller2012linear} or to 
the truncated SVD \cite{Hansen00thelcurve}, for some 
regularization parameter value: the significant difference is that the 
Bayesian framework considers that the regularization is an hypothesis with 
known parameter, while regularization methods produce the minimum 
required regularization parameter values on output. 
\end{frame}

%%%%%%%%%%%%%%%%%%%%%%%%%%%%%%%%%%%%%%%%%%%%%%%%%%%%%%%%%%%%%%%%%%%%%%%%%%%%%
\begin{frame}
\frametitle{Linear Gaussian Calibration}
We make the hypothesis that $\bh$ is linear with respect to
$\btheta$, i.e., for any
$\btheta\in\RR^p$, we have :

$$
h(\btheta) = \bh(\bmu) + J(\btheta-\bmu ),
$$

Let $J \in \RR^{n \times p}$ be the Jacobian matrix made of
the partial derivatives of $\bh$ with respect to
$\btheta$:

$$
J(\btheta) = \frac{\partial \bh (\btheta)}{\partial \btheta}.
$$

If the covariance matrix $B$ is positive definite, then the
Hessian matrix of the cost function is positive definite 
(no matter of the rank of $J$). 


Let $A$ be the matrix:

$$
A^{-1} = B^{-1} + J^T R^{-1} J.
$$



\end{frame}

%%%%%%%%%%%%%%%%%%%%%%%%%%%%%%%%%%%%%%%%%%%%%%%%%%%%%%%%%%%%%%%%%%%%%%%%%%%%%
\begin{frame}

We denote by $K$ the Kalman matrix:

$$
K = A J^T R^{-1}.
$$

The maximum of the posterior distribution of $\btheta$ given
the observations $\by$ is:

$$
\hat{\btheta} = \bmu + K (\by - \bh(\bmu)).
$$


It can be proved that (e.g. \cite{RasmussenWilliams}, p.9):

$$
   p(\btheta | \by) \propto 
   \exp\left(\frac{1}{2} (\btheta - \hat{\btheta})^T A^{-1} (\btheta - \hat{\btheta}) \right)
$$

for any $\btheta\in\RR^p$.

This implies:

$$
\hat{\btheta} \sim \mathcal{N}(\btheta,A)
$$
\end{frame}


%%%%%%%%%%%%%%%%%%%%%%%%%%%%%%%%%%%%%%%%%%%%%%%%%%%%%%%%%%%%%%%%%%%%%%%%%%%%%
\begin{frame}
\frametitle{Non Linear Gaussian Calibration : 3DVAR}

The cost function of the gaussian nonlinear calibration problem is :

$$
c(\btheta) = \frac{1}{2}\|\by-\bh(\btheta)\|^2_R 
   + \frac{1}{2}\|\btheta-\bmu \|^2_B
$$

for any $\btheta\in\RR^p$.

The goal of the non linear gaussian calibration is to find the value of
$\btheta$ which minimizes the cost function $c$. In
general, this involves using a nonlinear unconstrained optimization
solver.


\end{frame}


%%%%%%%%%%%%%%%%%%%%%%%%%%%%%%%%%%%%%%%%%%%%%%%%%%%%%%%%%%%%%%%%%%%%%%%%%%%%%
\section{General bayesian calibration}

%%%%%%%%%%%%%%%%%%%%%%%%%%%%%%%%%%%%%%%%%%%%%%%%%%%%%%%%%%%%%%%%%%%%%%%%%%%%%

\begin{frame}
\frametitle{General bayesian calibration}

If the full posterior distribution is to be known, the 
general bayesian framework may be used. 

This allows to:
\begin{itemize}
\item use whatever (proper) prior distribution (no need to restrict to gaussian 
prior),
\item use whatever computer code $\bh$ (no need to restrict to linear 
code).
\end{itemize}

We generally do not know the posterior distribution, but we can 
try to generate a sample from it. 

One method is to create a Monte-Carlo Markov Chain ; the classical algorithm is then the 
Metropolis-Hastings algorithm. 

However, sampling from the posterior distribution 
requires to generate a large sample size, which is impractical 
when the $\bh$ is costly. 

In this case, using a surrogate model is \emph{mandatory}. 

\end{frame}

%%%%%%%%%%%%%%%%%%%%%%%%%%%%%%%%%%%%%%%%%%%%%%%%%%%%%%%%%%%%%%%%%%%%%%%%%%%%%

\section{Conclusion}

%%%%%%%%%%%%%%%%%%%%%%%%%%%%%%%%%%%%%%%%%%%%%%%%%%%%%%%%%%%%%%%%%%%%%%%%%%%%%

\begin{frame}
\frametitle{Conclusion}

The following table presents a hierarchy of methods depending on the 
prior and the linearity of the function $\bh$.

\begin{center}
\begin{tabular}{lll}
                     & {\bf Linear}  & {\bf Non linear} \\
\hline
Without \emph{prior} & Linear        & Non Linear \\
                     & Least squares & Least Squares \\
\hline
\emph{Prior}: Gaussian & Linear Gaussian & 3DVAR \\
\hline
\emph{Prior}: & MCMC & MCMC \\
Non Gaussian &  &  \\
\hline
\end{tabular}
\end{center}

The previous methods define in-between solutions, 
\begin{itemize}
\item more complex than the least squares criterion, but more robust with 
respect to rank deficiency, 
\item less powerful than the MCMC simulation, but with less computational 
cost. 
\end{itemize}

\end{frame}

%%%%%%%%%%%%%%%%%%%%%%%%%%%%%%%%%%%%
\section{Bibliography}
\begin{frame}[allowframebreaks]
\frametitle{Bibliography}
\nocite{*}
\bibliographystyle{plain}
\bibliography{calibration}
\end{frame}
%%%%%%%%%%%%%%%%%%%%%%%%%%%%%%%%%%%%

\end{document}

